\documentclass[a4paper,10pt]{article}
\usepackage[T1]{fontenc}
\usepackage{amssymb}
\usepackage[portuguese]{babel}
\usepackage[utf8]{inputenc}
%\bibliographystyle{ieeetr}
\bibliography{topicos1}

%opening
\title{O paradoxo de Russel e suas contribuções para a Teoria dos Conjuntos}
\author{Francisco Rosa Dias de Miranda}

\begin{document}

\maketitle

\section{Introdução}

A Teoria dos Conjuntos tornou-se um dos mais importantes campos de estudo ao longo da história. Grandes matemáticos do século XX como Cantor, Hilbert, Von Neumann, Russell, Göedel entre muitos outros contribuíram para içá-la ao posto de teoria fundamentadora da matemática.\cite{rpm04}

Por ser de grande importância ainda nos dias de hoje, até mesmo algumas noções de o que podemos chamar de Teoria "Ingênua" dos Conjuntos\footnote{Diversos autores utilizam essa nomenclatura para referir-se a uma abordagem de Teoria dos Conjuntos que não a livre dos paradoxos descobertos ao longo da história, ocasionados por uma definição mais informal de conjunto.} estão presentes no currículo dos Ensinos Fundamental e Médio. 

Dentre os principais avanços em Teoria dos Conjuntos está o impacto que suas descobertas provocaram na forma em que axiomatizamos a matemática moderna. Contudo, tal axiomatização (ou tentativa de axiomatizá-la) não pode ser por si só livre de contradições e inconsistências. Mas tais descobertas só foram constatadas e provadas ao decorrer desse processo, como veremos neste texto. 

\section{O surgimento da Teoria dos Conjuntos}

O estudo de Teoria dos conjuntos foi iniciado por Georg Cantor em 1878. 
Este brilhante matemático introduziu o conceito de cardinalidade para conjuntos, que definiremos adiante.

Em matemática, a cardinalidade de um conjunto é uma medida de seu "tamanho". Seja $A$ um conjunto com $n$ elementos. A cardinalidade de $A$, ou simplesmente $\#A = n$.

Para conjuntos infinitos, uma forma de estudar sua cardinalidade é através de bijeções com os números naturais. Dizemos que um conjunto é \emph{enumerável} se é possível estabelecer uma correspondência biunívoca entre este conjunto e o conjunto dos números naturais, e \emph{não-enumerável} quando não é possível estabelecer tal relação.

Através desse argumento, Cantor foi capaz de provar a enumerabilidade do conjunto dos números racionais. E mais, ele também foi capaz de provar a não-enumerabilidade dos reais, estabelecendo assim que até mesmo o intervalo $[0,1]$ contém mais números que todo o conjunto dos naturais ou racionais.\footnote{As demonstrações da enumerabilidade de $\mathbb{Q}$, não-enumerabilidade de $\mathbb{R}$ e da densidade da reta estão disponíveis em \cite{histinfinito}.}

Essa descoberta revelou um fato estarrecedor: existem pelo menos dois tipos diferentes de infinito, o do conjunto dos números naturais e o do conjunto dos números reais.\cite{rpm04} Esse fato ficou conhecido como Hipótese do Continuum.

Logo após a sedimentação proposta por Cantor, Gottlob Frege (1884) fez uma tentativa de formalizar a matemática através de elementos presentes nessa recém-criada Teoria dos Conjuntos. Mas, antes mesmo de publicar o volume final de seu trabalho, Fredge recebeu uma carta de Bertrand Russell (1902), que havia descoberto uma inconsistência estutural no recém-criado sistema.

Tal fato gerou enorme consternação em Frege, conforme pode ser constatado em fragmento de sua resposta a Russell:
\begin{quote}
"Sua descoberta da contradição me causou uma enorme surpresa e, eu quase diria, consternação, já que ela abalou as bases nas quais eu pretendia erigir a aritmética. (...) Devo refletir aprofundadamente nesse assunto. Já que é muito grave que não apenas as fundações da minha aritmética, mas também as únicas fundações possíveis da aritmética, parecem se perder."\cite{carta}
\end{quote}

Adiante, explicaremos como o Paradoxo de Russell mudou completamente a base sobre a qual a Teoria dos Conjuntos erguia-se. Como veremos, tal formulação foi uma principais intuições a contribuir na concepção da matemática na forma em que hoje estudamos.

\section{Paradoxos e possíveis soluções}

Usar conjuntos para formalizar a estrutura da matemática consiste em considerar todos os objetos matemáticos como conjuntos. Tudo é conjunto.\cite{fajardo} Números, funções e relações são conjuntos, assim como os próprios elementos de um conjunto podem ser conjuntos.

Assim, genericamente, dada uma propriedade se forme o conjunto dos objetos que têm essa propriedade. \cite{conjconj}

Frege tentou, com ajuda dessa descrição, propor uma formalização da matemática em que a lógica e os conjuntos eram praticamente indissociáveis, que mais tarde começou a ser chamada de "ingênua", por conta Bertrand Russell ter encontrado inconsistências nessa formalização, através de seu famoso paradoxo:

Seja $X$ um conjunto, com os elementos definidos da seguinte forma:

$$\{{x : x \notin X}\}.$$ Tal definição leva a uma contradição, pois, para cada $x \in X \Leftrightarrow  x \notin X$.

Esses  paradoxos são  gerados  pela  flexibilidade  da própria  linguagem, assim sendo era necessário para a matemática construir um modelo que estabelecesse resultados importantes a partir da teoria sem gerar tais contradições.\cite{histinfinito}

Uma das primeiras tentativas, nesse sentido, foi proposta pelo próprio Russell com a sua Teoria dos Tipos. Outro pioneiro nesse campo foi David Hilbert em 1900, através seu programa de fundamentação da matemática.

Embora as ideias de Cantor sobre a Teoria dos Conjuntos tenham possibilitado uma construção inteiramente nova, ele não conseguiu provar um fato essencial, a já mencionada Hipótese do Continuum, basicamente a pergunta: existe um conjunto de cardinalidade maior que a do conjunto dos números naturais e menor que a dos reais?
 
Hilbert listou a Hipótese do Continuum como o primeiro dentre outros 23 problemas. O segundo deles era formalizar a Matemática, prová-la consistente e livre de contradições.

Formulações axiomáticas da Teoria dos Conjuntos logo precederam Russell e a mais conhecida delas hoje é chamada de ZFC (Zermelo-Fraenkel-Choice), que englobou as contribuições de Ernst Zermelo, Abraham Fraenkel e Thoralf Skolem entre os anos de 1908 e 1930.\cite{zfc}

As classificações de conjuntos informais e irrestritas da Teoria "Ingênua" foram substituídas por axiomas que buscavam acabar com as contradições. Ou seja, dados um conjunto $A$ e uma propriedade $P(x)$, existe um conjunto $M$ cujos elementos são os elementos de $A$ que satisfazem a propriedade $P(x)$.\cite{intro_tese}

Assim, a concepção de um modelo axiomático para a Teoria dos Conjuntos, que superava os paradoxos anteriores (embora não livre de novos), logo possibilitou diversos avanços para a Matemática. Uma das primeiras contribuições realizadas nesse sentido foi dada por Kurt Gödel, que, através de seus Teoremas de Incompletude, atacou os dois problemas de Hilbert já mencionados.

Não traremos os teoremas, mas para falar de suas implicações, uma delas foi de certa forma resolver o primeiro problema de Hilbert, provando que a negação da Hipótese do Contínuo não era provável em ZFC.\footnote{Paul Cohen (1963), através da técnica do \emph{forcing}, criada por ele para resolver proposições indecidíveis, que provou o afirmativo.}

O segundo teorema de Gödel diz, em grossas palavras, que qualquer prova de consistência da aritmética que utilize a própria aritmética para fazê-lo é inconsistente. Logo, é impossível alcançar o objetivo de Hilbert (embora alguns matemáticos acreditem que ele ainda esteja em aberto).

Os teoremas da Incompletude tiveram um grande papel em determinarmos até que grau podemos demonstrar a consistência de um sistema, de forma que é impossível a ZFC demonstrar sua própria consistência. O que podemos fazer é provar que aceitar o axioma da escolha é consistente com ZF (embora não aceitá-lo também o seja).

\section{Conclusão}

Desde Cantor surgiram muitas novas disciplinas matemáticas que se desenvolveram extensamente, como a Topologia, a Análise Real, a Teoria da Medida e Integração, a Teoria da Probabilidade, e outras mais. \cite{rpm04}

Há outros modelos em teorias dos conjuntos além de ZFC, por exemplo a NBG (von Neumann-Bernays-Godel set theory) e a MK (Morse-Kelly set theory), em que a aritmética de Peano também é consistente.

Existe até mesmo um modelo mais geral, a Teoria das Categorias, que generaliza conjuntos, grupos, espaços vetoriais, e até mesmo grafos como um único objeto matemático - as categorias. 

Em todas essas disciplinas a linguagem, a notação e os resultados da Teoria dos Conjuntos se revelaram instrumento natural de trabalho, a ponto de ser impossível conceber o desenvolvimento de toda essa Matemática sem a Teoria dos Conjuntos.

\begin{thebibliography}{topicos1}

\bibitem{rpm04} Geraldo Ávila (2000) Revista do Professor de Matemática, \textit{A teoria dos conjuntos e o ensino de Matemática} \textit {http://rpm.org.br/cdrpm/4/2.htm}

\bibitem{histinfinito} Christiano Otávio de Rezende Sena (2011), \textit{“Uma história sobre o infinito atual”},
\textit{https://repositorio.ufmg.br/bitstream/1843/BUBD-9ATLBH/1/monografia_christianootavio.pdf}

\bibitem{carta} Carta de Russell a Frege (16 de Junho de 1902). Disponível em http://nulfic.org/traducoes/gottlob-frege/carta-de-russell-a-frege-16-de-junho-de-1902/

\bibitem{fajardo} Rogério Augusto dos Santos Fajardo (2018), \textit{Teoria dos Conjuntos}, \textit{https://www.ime.usp.br/~fajardo/Conjuntos.pdf}

\bibitem{conjconj} Gilda Ferreira (2015) Gazeta de Matematica \textit{O Conjunto de Todos os Conjuntos Nao Existe}, \textit{https://webpages.ciencias.ulisboa.pt/~gmferreira/conjuntos.pdf}

\bibitem{intro_tese} Renan Maneli Mezabarba (2012) \textit{Uma Introdução à Teoria Axiomática dos Conjuntos} \textit{https://fernandobatista89.files.wordpress.com/2013/03/uma-introduc3a7c3a3o-c3a0-teoria-axiomc3a1tica-dos-conjuntos.pdf}


\bibitem{zfc} James T. Smith (2008) \textit{ZERMELO–FRAENKEL SET THEORY}, \textit{http://math.sfsu.edu/smith/Math800/Units/Zermelo-FraenkelSetTheory.pdf}


\end{thebibliography}
\end{document}
